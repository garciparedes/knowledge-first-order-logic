\documentclass[10pt, a4paper,spanish]{article}
\usepackage[utf8]{inputenc}

\usepackage{lipsum} % Package to generate dummy text throughout this template
\usepackage{varwidth}
\usepackage{hyperref}
\usepackage{graphicx}

\usepackage[T1]{fontenc} % Use 8-bit encoding that has 256 glyphs
\usepackage{microtype} % Slightly tweak font spacing for aesthetics

\usepackage[hmarginratio=1:1,top=32mm,columnsep=20pt]{geometry} % Document margins
\usepackage[hang, small,labelfont=bf,up,textfont=it,up]{caption} % Custom captions under/above floats in tables or figures
\usepackage{booktabs} % Horizontal rules in tables
\usepackage{float} % Required for tables and figures in the multi-column environment - they need to be placed in specific locations with the [H] (e.g. \begin{table}[H])
\usepackage{hyperref} % For hyperlinks in the PDF

\usepackage{lettrine} % The lettrine is the first enlarged letter at the beginning of the text
\usepackage{paralist} % Used for the compactitem environment which makes bullet points with less space between them

\usepackage{abstract} % Allows abstract customization
\renewcommand{\abstractnamefont}{\normalfont\bfseries} % Set the "Abstract" text to bold
\renewcommand{\abstracttextfont}{\normalfont\small\itshape} % Set the abstract itself to small italic text

\usepackage{titlesec} % Allows customization of titles
\renewcommand\thesection{\Roman{section}} % Roman numerals for the sections
\renewcommand\thesubsection{\Roman{subsection}} % Roman numerals for subsections
\titleformat{\section}[block]{\large\scshape\centering}{\thesection.}{1em}{} % Change the look of the section titles
\titleformat{\subsection}[block]{\large}{\thesubsection.}{1em}{} % Change the look of the section titles

\usepackage{fancyhdr} % Headers and footers
\pagestyle{fancy} % All pages have headers and footers
\fancyhead{} % Blank out the default header
\fancyfoot{} % Blank out the default footer
\fancyhead[C]{ \today $\bullet$ Entrega voluntaria sobre Lógica de Primer Orden} % Custom header text
\fancyfoot[RO,LE]{\thepage} % Custom footer text

%----------------------------------------------------------------------------------------
%	TITLE SECTION
%----------------------------------------------------------------------------------------

\title{\vspace{-15mm}\fontsize{24pt}{10pt}\selectfont\textbf{Entrega voluntaria sobre Lógica de Primer Orden}} % Article title

\author{Sergio García Prado}
\date{\today}

%----------------------------------------------------------------------------------------

\begin{document}

	\maketitle % Insert title
	\thispagestyle{fancy} % All pages have headers and footers


%----------------------------------------------------------------------------------------
%	TEXT
%----------------------------------------------------------------------------------------

    \section{?`Cuáles de las siguientes expresiones son substituciones validas?}

        \begin{itemize}

            \item
				$\{ f(x)/y \}$
				\newline
				Válida: Cumple las condiciones para ser una substitución válida.

			\item
				$\{ f(x)/x \}$
				\newline
				Inválida: No cumple la condición que restringe la aparición de la variable en el término.

			\item
				$\{ g(x)/y, y/u \}$
				\newline
				Inválida: No cumple la condición que restringe la aparición de la variable en el término, aunque este sea en otra ligadura.

		\end{itemize}

	\section{Obtener el unificador más general, umg, de los siguientes conjuntos de expresiones:}

		\begin{itemize}

			\item
				$\{ P(x, y), P(f(u), g(x)) \}$
				\newline
				UMG: $\{ f(u)/x, g(f(u))/y \}$
				\newline
				En el caso de la segunda substitución, tal y como indica el algoritmo de unificación, una vez que se substituye una variable, esta afecta a todas sus siguientes apariciones en la expresión.

			\item
				$\{ P(x, y), P(f(y), x) \}$
				\newline
				UMG: $\not\exists$
				\newline
				En este caso el unificador más general no existe debido a que no unifican, ya que la variable substituida no puede aparecer en el término ($ f(y)/y $ no es válido).

		\end{itemize}


\end{document}
